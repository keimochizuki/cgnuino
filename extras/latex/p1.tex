\hypertarget{p1_s1}{}\section{Before getting started}\label{p1_s1}
cgnuino is a library to help psychologists and neuroscientists to create behavioral tasks using Arduino boards. It provides several (what I hope are) useful functions when you create behavioral tasks with electrical inputs (e.\+g., button press, lever move, etc) and outputs (e.\+g., L\+ED light, beep sound, etc). However, right now before getting started, you might be even uncertain whether Arduino is the best option to implement your task. Thus I start with pros and cons of using Arduino in controlling behavioral tasks for psychological and cognitive neuroscience.\hypertarget{p1_ss1}{}\subsection{Pro\+: easy to program}\label{p1_ss1}
First of all, writing programs in Arduino system is definitely easy for beginners compared with other programming languages such as Python, Ruby, C\# and so on. Arduino uses classical C and C++ languages in writing program files, which are called as Sketches in this system. Drivers for Arduino boards are distributed with Arduino I\+DE, which is a useful and easy-\/to-\/handle software you can use to write your own Sketch. So if you have any experience in writing programs with other languages, it should be fairly easy to write Arduino Sketches. And if you are not familiar with programming languages, Arduino can be still a good starting point due to its simplicity.\hypertarget{p1_ss2}{}\subsection{Pro\+: convenient signal in/out (\+I\+O)}\label{p1_ss2}
Standard Arduino boards have general purpose in/out (G\+P\+IO) pins, with which you can read or put out high (5V) or low (0V) voltage signal. (Be careful that some Arduino boards use 3.\+3V for high digital signal instead of 5V.) Using these G\+P\+IO pins as digital-\/out, you are able to directly turn on and off electrical parts using DC 5V power supply (e.\+g., lighting a L\+ED), as well as to control instruments that have external triggering mechanism (e.\+g., applying predetermined set of electrical stimulation train at the rising edge of 5V input). Using G\+P\+IO pins as digital-\/in, you can easily monitor the participant\textquotesingle{}s button press or lever manipulation. Arduino boards often have analog input pins, and some also have analog output pins, with which you can read or put out arbitrary voltage ranging from 0V to 5V, instead of binary low or high voltage.

Since these pins are physically implemented on Arduino boards as either pin sockets or soldering holes, using them is very easy. On the other hand, if you use a programming language running on a standard PC to execute your behavioral task, using these digital and analog I\+Os is not so easy, and you will normally need A/D or D/A converter to allow your computer to communicate with external instruments.\hypertarget{p1_ss3}{}\subsection{Pro\+: low cost}\label{p1_ss3}
In general, Arduino boards are inexpensive. In addition, since Arduino is an open-\/source project, there are even cheaper Arduino-\/compatible boards from third parties (although I personally recommend official products). On writing programs with Arduino, you (of course) need a PC. However, it does not require any high performance like high-\/end C\+PU, massive R\+AM memory or gorgeous video card. Just a common PC with ordinary performance will do, with any of Linux, Mac or Windows operating system.\hypertarget{p1_ss4}{}\subsection{Con\+: no graphical display}\label{p1_ss4}
The biggest disadvantage of Arduino must be the absence of graphical display. Yes, you can find several liquid crystal displays and O\+L\+ED displays that can be used with Arduino, none of which will satisfy the use of graphical display to present visual stimuli in psychological experiments. If you want to use visual stimuli just for the purpose of, for example, instructing timing of actions and telling trial conditions to the participant, then L\+E\+Ds with different colors may be enough to do the job. However, if you wish to present visual stimuli in a normal sense for psychological experiments, then Arduino is not a suitable option for your task. In this case, you should choose other programming language working on a standard computer with graphical display. One of the popular options for such usage would be Expyriment library working on Python (which I prefer), and Psychophysics Toolbox working on Matlab (which I @\#\$\%!).\hypertarget{p1_ss5}{}\subsection{Con\+: data saving}\label{p1_ss5}
After compiling your code (i.\+e., Sketch) and sending it to the on-\/board chip, Arduino runs in a \char`\"{}stand alone\char`\"{} way. U\+SB connection is commonly used to power running Arduino, but is not necessary and can be thrown away if external DC supply is provided. When the power supply is out, Arduino stops running and all the temporal values of variables are flushed (except for a few exceptions). Therefore, to record the participant\textquotesingle{}s behavioral performance, you need certain external mechanics to save it on a non-\/volatile storage in some way.

One of the ways is to use U\+SB serial connection to write task informations from Arduino to PC as a text file. Arduino can easily emit text to a serial port (by {\ttfamily Serial.\+println} function) to show it on serial monitor application running on the connected PC. This is useful when you check the behavior of the program, especially during prototyping. However, if you use serial monitor with automatical saving utility, the same serial interaction mechanism can be used as a \char`\"{}write to a text file\char`\"{} function. I personally use this method for data saving and it is working quite well.

The other possibility for data saving is the usage of a SD (or nowadays more common micro-\/\+SD) card. Standard Arduino boards do not have SD card slot. But there are several breakout boards (small external boards that can be connected to Arduino) for SD card. Also, Arduino has a library named {\ttfamily SD} that provides a simple IO interfaces to SD card. Using these materials, you can save the task performance to a text file on SD card, which is later transported into your PC and analyzed. 